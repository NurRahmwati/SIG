\section{wms}
Web Map Service (WMS) memberikan informasi kepada pengguna internet oleh tata ruang peta gambar .Umumnya , yang tersimpan di dalam tata ruang data tersebut data vektor .Itu adalah panjang ayub untuk menciptakan data vektor peta gambar dari .Biaya untuk mengurangi waktu , maka kami memanfaatkan linux cluster .Ketika peta diminta , itu adalah suatu koordinat lingkup didefinisikan dengan xmin persegi panjang , ymin , xmax , ymax harus dispesifikasikan .Kami merancang beban keseimbangan menurut algoritma untuk membagi permintaan ke dalam beberapa sub-rectangles .Setiap sub-rectangle dikirim ke sebuah wms sub-maps node untuk menghasilkan sekumpulan .Peta yang dihasilkan akan merekonstruksi dengan semuanya ini sub-maps .Semua ini sub-maps dihasilkan di paralel , jadi makin sedikit waktu yang seluruh habis di memproduksi sebuah peta .Bagaimana untuk membagi menurut adalah kunci masalah permintaan .Pertama , kami menghadirkan metode untuk menghitung tingkat 2d bobot beban distribusi peta lingkup .Kedua , node beban kemampuan yang dievaluasi .Kemudian , penulis hadir metode menurut untuk membelah seluruh .Kertas ini juga membahas algoritma kinerja pelaksanaan .
